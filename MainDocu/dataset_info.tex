
\section{Dataset and Information About Attributes}
This section contains an overview of the dataset used for this paper. The manner of extraction is also discussed in this section. Attributes are defined as well. 

There are two datasets available for research and both are considered for the empirical study. The DNC Dataset and the JNC Dataset. Both datasets have been compiled and retrieved from Flurry Analytics, a commercial analytics tool used by Playlab Inc to track user's behavior on their commercial mobile games. Some attributes were retrieved from Google Play Developer Console such as user's ratings and daily crash reports. The DNC dataset refers to the Dragon Cubes game while the JNC dataset refers to the Jungle Cubes game. Both dataset are restricted to Android platforms only.

\ref{table:dataset_overview} shows the overview of the datasets used for the study.

\begin{table}
\centering
\caption{Overview of Dataset Used}
\label{table:dataset_overview}
\begin{tabular}{| C{2cm} | C{1cm} | C{1cm} | C{1cm} | C{1.5cm} |}
\hline 
Dataset Filename & Game Title & Total Downloads & Overall Rating & Timeline of Dataset \\ 
\hline 
DNC Dataset Android 0511-0911 & Dragon Cubes & 50,000 - 100,000 & 4.2 out of 5 & May 11, 2015 - September 11, 2015 \\ 
\hline 
JNC Dataset Android 0511-0911 & Jungle Cubes & 100,000 - 500,000 & 4.3 out of 5 & May 11, 2015 - September 11, 2015 \\ 
\hline 
\end{tabular}
\end{table} 

Both datasets spans on a similar timeline, that is May 11 - September 11,2015. A span of four months have been deemed sufficient for analysis and increasing the timespan no longer yields better results.

\section{Attribute Information}
This section discusses the attributes used for this study.  \ref{table:dataset_attributes} contains the definition of attributes found in the dataset. These attributes have been gathered and selected from Flurry Analytics and Google Play Developer Console. 

The initial selection of dataset has been manually performed by the researchers which they have deemed sufficient for analysis and have potential impact to the DAU value. On the succeeding section, we analyzed each variable and their correlation values to determine which variables have high relationship with the DAU value.

\begin{itemize}
\item Install Date - Each instance in the dataset is organized by install date. This refers to the gregorian calendar date wherein an application is installed.
\item Cohort Size - Refers to the total amount of users who have installed the application on the given install date.
\item Day X - This represents the retention of the application given a certain date and cohort size. Installation date becomes day 0. Retention rate is the percentage of returning users on a specified install date. For example, day 1 has 40.75\% retention and 1200 cohort size. Therefore, 40.75\% of users have managed to return on day 1 (489 users in cohort size)
\item CrashesANRDay1 - reality, crash reports come in a day after the specified install date. For example, May 11,2015 has 3 crash reports. This means that this value was only retrieved on May 12, 2015.This counts the total number of crashes and ANRs (application not responding) reports from the application. This has a negative impact for the user experience. In reality, crash reports come in a day after the specified install date. For example, May 11,2015 has 3 crash reports. This means that this value was only retrieved on May 12, 2015.
\item DailyAverageRating - This refers to the average rating by users who choose to rate the application (1 to 5, 5 being the highest) on a given date. Rating an application is not mandatory. This is a primary determination for virality. Similar to CrashesANRDay1, the tally comes in a day after the specified install date.
\item LevelPlayedEvents - Refers to the accumulated event tally that is triggered when a user plays a level on the application. This is triggered upon tap of the 'Play' button. This event is reported no matter the outcome of the level being played.
\item LevelSuccessEvents - Refers to the accumulated events that are triggered if a user successfully completes a level. This is triggered when the 'Win' screen is shown to the user.
\item LevelFailedEvents - Refers to the accumulated events that are triggered if a user fails a level. This is triggered when the 'Lose' screen is shown to the user.
\item Session - Refers to the total amount of play sessions on a given install date. A high value for session count on a given install date means that there are a lot of playthrough activity 
\item MKTExpenses - This is the total amount of marketing expenses, in USD, spent to advertise the game. Given an install date, the marketing expense normally determines the cohort size. A high marketing expense means more advertising channels have been used to target more potential users to install the game.
\item ActiveUsers - This refers to the total amount of unique users who spent considerable time in the game given a certain date. This refers to the "stickiness" of the application.This is one of the attributes essential for determining a game's success.
\item ActiveUsersDay7 - This is similar to the ActiveUsers variable but offset 7 days after the install date. This is the variable to be predicted. 
\end{itemize}

In reality, given a install date, and one would like to know how many daily active users would there be 7 days after, the following variables will be used: Cohort Size, Day 1, CrashesANRDay1, DailyAverageRating, LevelPlayedEvents, LevelSuccessEvents, LevelFailedEvents, Sessions, MKTExpenses, and ActiveUsers.

Note that some variables like DailyAverageRating and CrashesANRDay1, only becomes available a day after. In a practical scenario, one could make predictions by Day 2 since it is assumed that all variables are readily available.