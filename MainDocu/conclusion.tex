\section{Conclusion and Future Work}
We now attempt to verify if it is possible to predict \textbf{DAU-Day7}. Using Flurry Analytics and Google Play Developer Console to extract data and user behavior, we analyze if certain features affect \textbf{DAU-Day7} and if it is possible for such models be used to predict \textbf{DAU-Day7}.

We consider how features are frequently selected by using our proposed feature selection scheme. We see that on all our learning algorithms, those that relate to session activity (total number of sessions and session length) are selected across all our test cases\footnote{Total of 6 test cases; 3 learning algorithms applied for JNC and DNC}. This supports the findings of \cite{ref:predicting_player_churn} wherein session count and length has an effect on application virality.

\subsection{Marketing expenses, session count and session length}
We observed that marketing expenses has been selected on all cases. Marketing expenses for JNC has high correlation with \textbf{DAU-Day7} (0.73). While \textit{MKTExpenses} are also being selected as a contributing feature for DNC, it yields a low correlation value when we analyzed the dataset. This supports the speculation of the marketing team that DNC does not have enough traction or "stickiness" wherein users who get to install the game leave before Day 7. Based from our study, we propose a finding that \textit{MKTExpenses} gets high correlation with \textbf{DAU-Day7} once the game has enough content to keep users engaged. To make engagement factors high, business decisions should improve the outcome of session length and number of sessions (\textit{Sessions} and \textit{MedianSessionSeconds} should have strong correlation with \textbf{DAU-Day7}). Once those are set, increasing marketing expenses(to increase social reach or promote the game), will also increase the potential of gaining additional daily active users. We propose to the readers to further verify this claim as future work.

\subsection{Factors in game design}
We observe that relying on gathered data and game events, JNC performed better and has a better accuracy on most of the training models we proposed. Using the same features for DNC, we see that it performed poorly in general. We therefore see that the success of a game also contributes to better data and discover more patterns than if a game did not really do well in the market. We deduce that DNC has underlying problems in how the game was designed. None of the features yield high correlation value with \textbf{DAU-Day7}. Events and patterns from the data are not observable. 

We propose as future work that such games like DNC should also have a concrete model to quantify the engagement factor of the game. Inferring the fun factor of a game may also be modelled by gathering user sentiment or initial feedback. Relying on events that only consider session length, triggered events and marketing expenses may not fully capture such cases like this.

\subsection{Considering quality of the game}
We see that we yield a somewhat high correlation value between \textit{CrashesANRDay1} and \textbf{DAU-Day7}. However, this is misleading; more crashes means more DAU-Day7 which should not be the case. We only see this as a probabilistic event wherein such events occuring is somewhat proportional to how many users are active. We should see such negative events that drive the \textbf{DAU-Day7} down. We therefore propose as future work that software quality be measured for games as it plays an important role in properly predicting user virality. Consider quality-of-life features in the game that makes it easy for the users to understand the mechanics. Some games tend to have sharp learning curves which makes them quit the game early.

To conclude our study, \textbf{DAU-Day7} can be predicted using our proposed method for successful games. JNC yields the lowest error rate which makes it ideal for our learning algorithms proposed in this paper. On the case of DNC, we conclude that it does not perform as expected on JNC and may not be practical to use such model. We propose that games wherein cases are similar to DNC, one should consider taking into consideration factors in game design and quality of the game. Out of all the learning algorithms tackled, we recommend using m5Base as the model to be used for practical application due to how well it covered different scenario in the data.