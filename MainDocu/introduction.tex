% Contains the introduction
%
%
%
\section{INTRODUCTION}
Free-to-play mobile games follow a common business model. It is a necessity that these games have a healthy community to drive the virality of the game. One of the key measures for determining the success of a mobile game is by using Daily Active Users (DAU) as a metric. This refers to the total amount of unique users who spent considerable time in the game given a certain date. This refers to the "stickiness" of the application. A high value for DAU indicates that there is much activity and demand for the mobile game. It is essential for game development companies that develop mobile games that follows a free-to-play model, to keep the DAU value as high as possible.  This paper attempts to predict DAU value for two commercial match-3 mobile games, Jungle Cubes and Dragon Cubes, that are currently owned by Playlab Inc.

Companies that follow a free-to-play business model use various analytics tools to track user's behavior and events triggered in their applications. In this study, Playlab Inc., uses Flurry Analytics and Google Play Developer Console to collect user data. One of the attributes capable of being tracked is the DAU.

In this study, we present a hypothesis that the DAU value is driven by other attributes or events that causes a user to use the application (or in this case, play for a considerable time) extensively. Such attributes considered are discussed in the succeeding sections.

Business decisions for a mobile game are executed when there is sufficient user data collected. One of the major problems encountered by companies such as Playlab Inc., is that major releases for the game don't immediately provide significant user data. It would be appropriate to provide a forecasting technique to determine if there would be a significant user activity on succeeding days. In a practical scenario, this paper aims to answer the question; given a certain day with these collected attributes, how high will the DAU value be X days after? In our experiments, we specifically attempt to predict the Daily Active Users value at Day 7.
